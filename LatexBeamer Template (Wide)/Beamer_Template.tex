\documentclass[xcolor={usenames,dvipsnames}, 
	hyperref={
	colorlinks=false, 						% Internetseiten werden farblich hervorgehoben
	linkcolor=black, 						% Farbe f�r interne Referenzen
	urlcolor=black,							% Farbe f�r Links auf Webseiten
	citecolor=black,						% Farbe f�r Zitate \cite{<bibtexid>}
	pdfpagelabels=false,
	%pdfauthor={}, 
	%pdftitle={}				% Verewigt Author und Titel in PDF-Informationen
	},
	ignorenonframetext,			% Keine Nummerierung auf erster Folie
	compress,					% Minimize navigation bars
	aspectratio=169
]{beamer}
%===== Formatierungs-Packages =====
	\usepackage[T1]{fontenc}				% fontenc und inputenc erm�glichen
	\usepackage[latin1]{inputenc}			% Silbentrennung und Eingabe 
	%\usepackage[utf8]{inputenc}
	\usepackage{lmodern}					% Schrift wirkt (in pdf-Ausgabe) flie�ender
  					

%===== Sprach-Packages =====
	%\usepackage[ngerman]{babel}			% Babel f�r diverse Sprachanpassungen, 
  										% z.B. Anf�hrungszeichen
%===== Interne Latex-Packages =====
%	\usepackage{fixltx2e} 				% Verbessert einige Kernkompetenzen von LaTeX2e
  
%===== Mathe-Packages ======
	\usepackage{amsmath, amssymb, amsfonts} 	% Mathematische Features der American Mathematical
	\usepackage{cancel}
	\usepackage[output-decimal-marker={,},  	% Deutsche Dezimaltrennung mit Komma
		separate-uncertainty = true,		  	% Fehlerangabe: \SI{3(2)}{\tesla}
		per-mode=fraction,					  	% Einheiten als Bruch darstellen
		exponent-product = \cdot,			  	% Exponentialschreibweise mit Malzeichen \SI{3e8}{\tesla}
		%math-ohm,
		range-phrase = -					% Option f�r Bereichsangabe \SIrange{3}{4}{\tesla}
		]{siunitx} 						  	
 								% Elementar wichtig f�r Einheiten \siunit{3}{\milli\meter}					% \unit{\tesla}, \num{<Zahl>}
  
%===== Grafik/Tabellen-Packages ======
	\usepackage{xcolor}						% Erlaubt das Verwenden von Farben
	\usepackage{graphicx} 					% Erlaub das Einbinden von Bildern
	\usepackage{multirow} 					% Erlaubt multicolumn{3}{c}{Bla}
	\usepackage{rotating} 					% Erlaubt sidewaystable-Umgebung
	\usepackage{subfig}
	%\usepackage[miktex,subfolder,siunitx]{gnuplottex} % Gnuplot in Latex
  
%===== bibliography =====
	\usepackage[numbers,square]{natbib} 	%Einbinden der Bibliothek mit "[1]" Zitat
  
%===== Sonstiges ======
	\usepackage{url}						% Erlaubt das Einbinden einer URL
	\usepackage{pdfpages}					% \includepdf{<file>.pdf} wird verf�gbar
	\usepackage[flushmargin,				% Fu�noten b�ndig mit Seitenr�ndern
		hang,									% H�ngender Zeilenumbruch bei Fu�noten 
		bottom]{footmisc}						% Zwingt Fu�noten ans Ende der Seite
	\usepackage{hyperref} 	% Hyperref-Package verlinkt interne Referenzen
	\usepackage{lipsum} 	% F�r Testzecke \lipsum[1]
  
%===== Beamer template - Spezifikationen ======
	\usepackage{beamerthemetexsx}
	\setbeamertemplate{mini frames}[box]
	%\renewcommand{\bibsection}{\section{Literature}}
%\setbeameroption{show notes on second screen=right}

\usepackage{appendixnumberbeamer}


%==== Informationen ====
	
	\title{Titel} 
	\subtitle[Optional Subtitle]{Subtitel}
	\author{David Osten} 
	\institute{Topic/Lecture Series}
	\date{xx/xx/2022}

\begin{document}

\begin{frame}[plain,noframenumbering]  %Keine Fu�zeile auf erster Seite und keine Nummerierung
	\titlepage
\end{frame}
	
\addtocounter{framenumber}{-1} %Titelseite wird nicht gez�hlt im Counter

%\setbeamertemplate{footline}{authorinstitutetitle} %Aktiviere Fu�zeile

%---------------------------------------------------------------------



%---------------------------------------------------------------------


\begin{frame}
	\frametitle{Motivation}
	\begin{itemize}
		\item 
		\item 
	\end{itemize}
\end{frame}

\begin{frame}
	\frametitle{Preview of Findings}
	Key Take-aways:
	\begin{itemize}
		\item 
		\item 
	\end{itemize}
\end{frame}

\begin{frame}
	\frametitle{Table of Contents}
	\tableofcontents
\end{frame}


\section{Section1}
\begin{frame}
	\frametitle{Frame1}
	\begin{itemize}
	\item 
	\item 
	\item
	\item 
	\item 
\end{itemize}

\end{frame}


\begin{frame}
\frametitle{Frame2}

\begin{itemize}
	\item 
	\item
	\item 
\end{itemize}

\end{frame}


%---------------------------------------------------------------------
\section{Section2}

\begin{frame}
\frametitle{Frame3}

\begin{itemize}
	\item 
	\item 
	\item
	\item 
	\item 
	\item 
	\item 
\end{itemize}
\end{frame}

\begin{frame}
	\frametitle{Frame4}
	%\label{Information Structure}
	\begin{itemize}
		\item  \hyperlink{Label}{\beamerbutton{more}}
		\item 
	\end{itemize}

\end{frame}



\begin{frame}
\frametitle{Frame5}
\begin{itemize} 
	\item 
	\item 
\end{itemize}
\ \\

\begin{equation*}
	1
\end{equation*}
\ \\

\begin{equation*}
1
\end{equation*}
\end{frame}

\begin{frame}
	\frametitle{}
	
\end{frame}

\section{Section3}
\begin{frame}
	\frametitle{Frame6}

	\begin{enumerate} 
		\item 
		\item 
		\item 
		\item 
	\end{enumerate}
\end{frame}

\begin{frame}
	\frametitle{Frame7}
	
	\begin{equation*}
		1
	\end{equation*}

\end{frame}

\begin{frame}
	\frametitle{Frame8}
	
	
	\begin{block}{}
	1
	\end{block}
	
\end{frame}

%---------------------------------------------------------------------
\section{Section4}
\begin{frame}
\frametitle{Frame9}


\begin{itemize}
\item
\item 
\end{itemize}

\end{frame}


\begin{frame}
\frametitle{Frame10}
\begin{itemize}
	\item 
	\item
	\item 
	\item
	\item 
	\item 
\end{itemize}
\end{frame}

\begin{frame}
\frametitle{Frame11}

\end{frame}


%---------------------------------------------------------------------




% \begin{frame}
%	\frametitle{}
%	\begin{center}
%		\includegraphics[scale = 0.45]{Name.PNG}
%	\end{center}
%\end{frame}

%---------------------------------------------------------------------


\begin{frame}[plain,noframenumbering] 
\begin{center}
	\Huge{Titel}
\end{center}
\ \\
\begin{center}
	\Large{Subtitel}
\end{center}
\end{frame}
%---------------------------------------------------------------------


%%%%%% Appendix

\begin{frame}
	\label{Label1}
	\frametitle{}
	\hyperlink{Label2}{\beamerbutton{back}}
	
\end{frame}


%---------------------------------------------------------------------

%\section{Literature}



%---------------------------------------------------------------------




\end{document}